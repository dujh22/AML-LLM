\chapter*{序}

近年来,随着自然语言处理任务复杂度的不断提高,传统的机器学习算法已难以满足实际需求。在数据、算力和人工智能快速发展的背景下,深度学习已成为推动技术进步的核心力量。深度学习不仅在图像和语音处理等传统领域取得了重大突破,还在大语言模型领域展现出巨大的应用潜力。本书将系统介绍深度学习及其在大语言模型中的应用,帮助读者全面理解和掌握这项前沿技术。

通过学习深度学习和大语言模型的理论与应用,读者将能够有效应对不同复杂环境和任务,发展具有人脑模拟功能的智能模型。这不仅是数据挖掘和信息处理的核心技术,还已成为计算机学科研究生必备的技能。本书通过理论讲解和实践操作,使读者不仅能够熟练掌握深度学习和大语言模型的先进理论知识,还能够将其应用于实际项目和新一代智能信息处理系统中。

本书由清华大学唐杰教授和杜晋华等博士生团队共同编写,旨在为读者提供深入了解深度学习和大语言模型的机会。本书内容全面,涵盖了深度学习和大语言模型的基本理论、算法实现和应用案例,适合作为高等院校计算机相关专业的教材,也可作为从事相关领域的科研人员和工程技术人员的参考书。

本书的编写得到了清华大学计算机系高级机器学习课程组的大力支持,在此表示感谢。同时,感谢所有为本书的编写提供帮助的专家和学者,他们的贡献使得本书得以顺利完成。

具体第一章的作者包括:唐杰、杜晋华、顾晓涛、隋元培、张远达、黄茜瑛。
第二章的作者包括:唐杰、杜晋华、杨超群、吴彦辰。



\begin{flushright}
    杜晋华\\
    2025年7月30日
\end{flushright}