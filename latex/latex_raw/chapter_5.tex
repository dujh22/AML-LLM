\chapter{交叉引用与其他}
\section{交叉引用}
在上一章我们强调了表格与图片需要在页面中自动寻找合适的位置,而不是固定在某个位置,那么我们需要引用表格内容时就不能用“如下表”类似的话语,而需要用到交叉引用,也就是“如图1”类似的文字。

细心的读者可以发现在浮动体环境内作者均加有\verb|\label{key}|的语句,这是为了后文方便引用图片内容而写,每个图片、表格的$key$值必须唯一。
在做引用的时候需要用到命令\verb|\ref{label}|,$label$处填写唯一的$key$值,这样就可以做到交叉引用,更方便的是在pdf阅读时,可以通过单击索引小标来定位到该图片处,如这里引用之前的图片就可以写如图$^{[\ref{fig1}]}$。

除了浮动体可以做交叉引用,公式也是可以做交叉引用的,这里引用文章出现的第一个公式可以写如公式$^{[\ref{equ1}]}$。

需要注意的是,目录与交叉引用需要至少编译两次,也就是点两次编译按钮。

由于本科实验报告中不要求添加参考文献,这里不做多于的叙述,详细的在lshort中也有叙述。
\section{其他}
这里将会叙述一些其他的知识,有的是在作者平时写作中遇到的,有的是模板写作中的问题,本章节会即时更新。

按照学校模板要求,在封面字体需为仿宋GB2312 加粗,事实上,字体的加粗并不想word中强制加粗(会出现很多很多很多错误),而是用其他粗体的字体来代替,因此本文由于版权限制,使用黑体代替格式中要求的字体。

在使用matlab绘图时,可以在图例、坐标轴等地方利用\LaTeX 公式写出分式等符号,但是matlab中只允许一部分\LaTeX 代码,并不是全部的数学公式代码都可以用。

\LaTeX 有相当多的宏包用于不同的环境,神经网络、化学、生物等等学科的图都可以在宏包中找到。